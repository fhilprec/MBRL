%% 
%%  This is file `URbeamer-DEMO.tex' version 2.0 (2017/04/05),
%%	it is part of
%%  urcls -- Corporate Design for the University of Regensburg
%% ----------------------------------------------------------------------------
%%
%%  Copyright (C) 2016--2017 by Marei Peischl <TeX@mareipeischl.de>
%%
%% ----------------------------------------------------------------------------
%%  License information
%% ----------------------------------------------------------------------------
%%
%% This work may be distributed and/or modified under the
%% conditions of the LaTeX Project Public License, either version 1.3
%% of this license or (at your option) any later version.
%% The latest version of this license is in
%%   http://www.latex-project.org/lppl.txt
%% and version 1.3 or later is part of all distributions of LaTeX
%% version 2005/12/01 or later.
%%
%% This work has the LPPL maintenance status `maintained'.
%%
%% The Current Maintainer of this work is Marei Peischl.
%%
%% ============================================================================
%%
%%  Dieses Werk darf nach den Bedingungen der LaTeX Project Public Lizenz
%%  in der Version 1.3c, verteilt und/oder verändert werden. Die aktuelle
%%  Version dieser Lizenz ist http://www.latex-project.org/lppl.txt und
%%  Version 1.3c oder neuer ist Teil aller LaTeX-Distributionen ab 2005/12/01. 
%%  Dieses Werk hat den LPPL-Verwaltungs-Status "maintained". 
%%  Die Verwaltung liegt aktuell bei der Autorin, Marei Peischl.
%%
%% ----------------------------------------------------------------------------
%%  End of license information
%% ----------------------------------------------------------------------------
%%
%%Diese Datei dient als Demonstration zu Umsetzung der Gestaltungsrichtlinien zum Corporate Design der Universität Regensburg in LaTeX.
%%Die Dateien werden in ihrer aktuellen Form bereitgestellt, allerdings übernimmt die Autorin keinerlei Verantwortung für die Verwendung.
%%Bei Fragen, Wünschen oder Anregungen freue ich mich über eine Email: TeX@mareipeischl.de
%%Selbiges gilt, wenn Sie daran interessiert sind, die Weiterentwicklung sowie die Verbesserung der Dokumentation zu unterstützen.
%%
\documentclass[english,%Dokumentensprache
	aspectratio=169,%Seitenverhältnis von 16:9
	colors={rz,faculties},%Farbbalken 
	framenumber=true,%Foliennummer in der Kopfzeile
	externalize=true,
	]{URbeamer}
	


\usepackage{iftex}%automatische Auswahl des richtigen Fontloaders und der Eingabekodierung
%Es liefert das Makro \ifPDFTeX. Die Abfragen können entfernt werden, wenn nur eine bestimmte Variante verwendet wird.

\ifPDFTeX%falls mit pdfLaTeX kompiliert wird
	%Eingabekodierung (nur notwendig bie pdflatex)
	\usepackage[utf8]{inputenc}
	%Für die Hausschriftart der Universität Regensburg, falls installiert:
	%weitere Informationen unter: http://www.physik.uni-regensburg.de/studium/edverg/latex/files/cd/cd.phtml
	\usepackage[T1]{fontenc}
	\usepackage{frutigernext}
\else%falls mit Lua oder XeLaTeX kompiliert wird
	%Für die Hausschriftart der Universität Regensburg (zusätzliche Installation notwendig)
	%weitere Informationen unter: http://www.physik.uni-regensburg.de/studium/edverg/latex/files/cd/cd.phtml
	\usepackage{fontspec}
	\setmainfont{Frutiger Next LT W1G}
\fi

\usepackage{babel}


%-------------------------------------------------------------------------------------------------------------
%Einstellungen für den Inhalt des Dokumentes. Im Allgemeinen nicht notwendig!                        
\usepackage{hologo}           
\newcounter{iterator}  
\newcommand*\code[1]{\texttt{#1}}
%-------------------------------------------------------------------------------------------------------------       
                            
                            
%Datenübergabe für die Titelfolie
%Die Makros \and und \inst können wie gewohnt verwendet werden
\title{\LaTeX-beamer}
\subtitle{Im Corporate Design\\\hspace*{\fill} der Universität Regensburg}
\department{\LaTeX-Kurse und  Support}
%\institute[]{Institut für Experimentelle und Angewandte Physik}
\date{3. April 2017}
%\chair[LS für \ldots]{Lehrstuhl für \ldots}
\author[Marei Peischl]{Marei Peischl\URbeamerIgnoreMeta{ (\url{TeX@mareipeischl.de})}}%Die Kurzversion wird in der Kopfzeile eingetragen
                       
                     
                        
\begin{document}
	
\frame[plain]{\titlepage}

\begin{frame}{Hausschriftart: Frutiger Next LT W1G}
	Das Rechenzentrum stellt für die Installation der Hausschriftart auf Dienstrechnern einen \href{http://www.uni-regensburg.de/rechenzentrum/software/softwarekatalog/produktdetails/index.html?product_hash=cc47a29792efce83538cdf5660de6f5d}{Installer} im Softwarekatalog an. 

	\medskip
	Der Installer enthält nur die OpenType-Variante. Daher ist es für die Benutzung notwendig mit \hologo{XeLaTeX} oder \hologo{LuaLaTeX} zu kompilieren. Darüber hinaus ist der \LaTeX-Compiler frei wählbar (vgl. auch Präambel des Quellcodes zu diesem Dokument).
\end{frame}

\begin{frame}{Titel \& Autorendaten}
	Die Daten für die Titelseite und die Kopfzeile werden mithilfe der üblichen Makros übergeben (\code{\textbackslash{}title}, \code{\textbackslash{}author}, \code{\textbackslash{}date}, \code{\textbackslash{}institute}).
	
	\smallskip
	Zusätzlich liefert URbeamer noch weiter Möglichkeiten zur Angabe einer Fakultät/Einrichtung (\code{\textbackslash{}department}) oder eines Lehrstuhles (\code{\textbackslash{}chair}). Diese Unterscheidung ist aufgrund der Gestaltungsrichtlinien notwendig.

	\smallskip	
	Alle diese Makros erlauben es, wie gewohnt, eine Kurzversion als optionales Argument zu übergeben.
	
	\medskip
	\small
	\emph{Bemerkung:} Wenn das Feld \code{\textbackslash{}department} leer ist, wird es mit dem Feld \code{\textbackslash{}institute} vertauscht. Dies dient dazu, dass alte Dateien nach wie vor das gleiche Ergebnis in der Ausgabe erzielen.
	
	\smallskip
	\normalsize
	Die Metadaten werden automatisch durch die Titeldaten erzeugt. Zusätzliche Angaben, die nicht in den Metadaten erscheinen sollen, können mit \code{\textbackslash{}URbeamerIgnoreMeta\{\}} entsprechend gesetzt werden. Ein Beispiel hierfür ist die Emailadresse auf der Titelfolie dieses Dokuments.
\end{frame}

\begin{frame}{Titelseite}
	Da die Titelseite des Layouts die gesamte Folie ausfüllt ist als Option der Folie \code{plain} zu setzen:
	
	\begin{block}{Erzeugung der Titelseite}
	 \ttfamily\textbackslash{}frame[plain]\{\textbackslash{}titlepage\}
	 \end{block}
\end{frame}

\begin{frame}{Auswahl der Fakultätsfarben}
Die Fakultätsfarben werden durch Angabe der zugehörigen Dokumentenklassenoption ausgewählt. Die Neue Version unterstützt zudem die Angabe eine Farbliste in der Form 
\begin{block}{\ttfamily colors=\{Fakultät1,Fakultät2,Fakultät3\}}
z.\,B.: 
\ttfamily colors=\{ph,slk,rz\}
\end{block}
Diese Variante sorgt für eine bessere Möglichkeit der Gliederung bei den Dokumentenklassenoptionen. Die Reihenfolge der Angaben ist dabei nicht relevant. Die Sortierung erfolgt automatisch nach den Richtlinien zum Corporate Design.

Die Fakultäten und Einrichtungen können über die entsprechenden Kürzel übergeben werden. Eine entsprechende Legende findet sich auf den nächsten Seiten.
\end{frame}


\begin{frame}{Auflistung der Fakultätskürzel}
	\setcounter{iterator}{3}
	\begin{tabular}{>{\stepcounter{iterator}\cellcolor{UR@color@\theiterator}}p{7.5mm}p{\dimexpr\linewidth-7.5mm-3\tabcolsep\relax}@{}}
		rw&Fakultät für Rechtswissenschaft\\
		ww&Fakultät für Wirtschaftswissenschaften\\
		kt&Fakultät für katholische Theologie\\
		pkgg&Fakultät für Philosophie, Kunst-, Geschichts- und Gesellschaftswissenschaften\\
		pps&Fakultät für Psychologie, Pädagogik und Sportwissenschaft\\
		slk&Fakultät für Sprach-, Literatur- und Kulturwissenschaften\\
		bvm&Fakultät für Biologie und vorklinische Medizin\\
		mat&Fakultät für Mathematik\\
		ph&Fakultät für Physik\\
		chp&Fakultät für Chemie und Pharmazie\\
		med&Fakultät für Medizin
	\end{tabular}
\end{frame}

\begin{frame}{Liste der Kürzel für zentrale Einrichtungen}
\setcounter{iterator}{0}
\begin{tabular}{>{\stepcounter{iterator}\strut\color{white}\cellcolor{UR@color@\theiterator}}p{7.5mm}p{\dimexpr\linewidth-7.5mm-3\tabcolsep\relax}@{}}
lov&Leitung, Organe, Verwaltung\\
ffg&Chancengleicheit und Familie\\
asz&Service-Einrichtungen für Studierende\\
\noalign{\setcounter{iterator}{14}}
ub&Universitätsbibliothek\\
zsk&Zentrum für Sprache und Kommunikation\\
eur&Europaeum (Ost-West-Zentrum)\\
zhw&Zentrum für Hochschul- und Wissenschaftsdidaktik\\
rul&Regensburg Universitätszentrum für Lehrerbildung\\
zfw&Zentrum für Weiterbildung\\
spo&Sportzentrum \\
rz&Rechenzentrum\\
\end{tabular}
\end{frame}


\begin{frame}{Weitere Farboptionen und Farbkombinationen}
	
\begin{tabular}{@{}p{1.5cm}p{\dimexpr.7\linewidth-1.5cm-5\tabcolsep\relax}p{.3\linewidth}}
all&alle Einrichtungen&\URrule{lov,ffg,asz,rw,ww,kt,pkgg,pps,slk,bvm,mat,ph,chp,med,ub,zsk,eur,zhw,rul,zfw,spo,rz}{\linewidth}{5mm}\\
faculties&alle Fakultäten&\URrule{rw,ww,kt,pkgg,pps,slk,bvm,mat,ph,chp,med}{\linewidth}{5mm}\\
fsimphy&Fachschaft Mathe-Physik&\URrule{fsimphy}{\linewidth}{5mm}\\
\end{tabular}
\end{frame}
\begin{frame}{Entwurfsmodus}
Die \code{draft}-Option ersetzt den die farbigen Streifen durch graue Boxen, die ggf. eine Beschriftung des Templates enthalten.
Dies beschleunigt den Kompilierungsvorgang.

\bigskip
\small
Analog zu den anderen Optionen, kann diese Option Wahrheitswerte (\code{true}/\code{false}) verarbeiten. Wird kein Wert angegeben, so wird \code{true} als Defaultwert verwendet.
\end{frame}
\begin{frame}{Einmaliges Kompilieren genügt für die Farbbalken}
Die Implementierung der farbigen Streifen wurde mithilfe des Paketes URrules verbessert und ausgelagert. Somit ist es für die Streifenpositionierung und Farbanpassung ab dieser Version nicht mehr nötig mehrfach zu Kompilieren.
\end{frame}

\begin{frame}{TikZ-externalize-Funktion}
	Die Klassenoption externalize unterstützt die TikZ-Funktion \glqq{}externalize\grqq. Dazu muss URbeamer mit der entsprechenden Option geladen werden und anschließend die Ausgabe der pdf-Dateien mit \code{\textbackslash{}tiktexternalize} aktiviert werden. Für die Nutzung dieser Funktion muss \code{pdflatex} mit  der Option \code{-shell-escape} ausgeführt werden.
\end{frame}

\part{In dieser Version neue zusätzliche Funktionen \& Optionen}

\begin{frame}{Key-Value-Struktur für die Optionsverarbeitung}
	Die Optionsverarbeitung wurde auf eine Key-Value-Struktur umgestellt. Das bedeutet, dass sämtliche Optionen nun in der Form \textit{Schlüssel}\code{=}\textit{Wert} übergeben werden können. Wird kein Wert angegeben wird in der Regel ein Säumniswert verwendet. Die alte Variante ist somit über die Säumniswertvariante nach wie vor möglich.
\end{frame}

\begin{frame}{Erweiterung der Optionen}
	Um eine besser Übersicht bei der Options
\end{frame}

\begin{frame}{Offizielles Layout auf Basis der PowerPoint-Vorlage}
\small
Das Corporate Design sieht in der PowerPoint-Vorlage vor, dass die Einrichtung in Versalien gesetzt wird.(\code{depcaps=true})

\smallskip
Zudem wird dort das Logo mit Textmarke verwendet. (\code{logotext=true})

\smallskip
Die Größe des linken Seitenrandes ist in der PowerPoint-Vorlage bündig mit dem grauen Balken der Kopfzeile. (\code{alignwithbar=true})

\medskip
Um eine flexiblere Anpassung zu gewährleisten wurden die Standardeinstellungen für diese Dokumentenklasse entgegen dieser Vorgaben gewählt.
(Beispiel: Im Seitenformat 16:9 ist die Textmarke des Logos nach den Vorgaben deutlich zu klein)

\smallskip
Möchte man dennoch die offiziellen Vorgaben benutzen, existiert die Option \code{cdlayout=\textit{Wahrheitswert}}.
\end{frame}



\end{document} 